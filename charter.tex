\documentclass[11pt]{charter}

% El títulos de la memoria, se usa en la carátula y se puede usar el cualquier lugar del documento con el comando \ttitle
\titulo{Detector de descargas parciales} 

% Nombre del posgrado, se usa en la carátula y se puede usar el cualquier lugar del documento con el comando \degreename
\posgrado{Carrera de Especialización en Sistemas Embebidos} 
%\posgrado{Carrera de Especialización en Internet de las Cosas} 
%\posgrado{Carrera de Especialización en Intelegencia Artificial}
%\posgrado{Maestría en Sistemas Embebidos} 
%\posgrado{Maestría en Internet de las cosas}

% Tu nombre, se puede usar el cualquier lugar del documento con el comando \authorname
\autor{Pablo Severini} 

% El nombre del director y co-director, se puede usar el cualquier lugar del documento con el comando \supname y \cosupname y \pertesupname y \pertecosupname
\director{Marcos Maillot}
\pertenenciaDirector{UTN FRP} 
% FIXME:NO IMPLEMENTADO EL CODIRECTOR ni su pertenencia
\codirector{Cristian Bonini} % si queda vacio no se deberíá incluir 
\pertenenciaCoDirector{UTN FRP}

% Nombre del cliente, quien va a aprobar los resultados del proyecto, se puede usar con el comando \clientename y \empclientename
\cliente{Nombre del cliente}
\empresaCliente{Empresa del cliente}

% Nombre y pertenencia de los jurados, se pueden usar el cualquier lugar del documento con el comando \jurunoname, \jurdosname y \jurtresname y \perteunoname, \pertedosname y \pertetresname.
\juradoUno{Nombre y Apellido (1)}
\pertenenciaJurUno{pertenencia (1)} 
\juradoDos{Nombre y Apellido (2)}
\pertenenciaJurDos{pertenencia (2)}
\juradoTres{Nombre y Apellido (3)}
\pertenenciaJurTres{pertenencia (3)}
 
\fechaINICIO{22 de junio de 2020}		%Fecha de inicio de la cursada de GdP \fechaInicioName
\fechaFINALPlanificacion{22 de Agosto de 2020} 	%Fecha de final de cursada de GdP
\fechaFINALTrabajo{22 de diciembre de 2020}		%Fecha de defensa pública del trabajo final


\begin{document}

\maketitle
\thispagestyle{empty}
\pagebreak


\thispagestyle{empty}
{\setlength{\parskip}{0pt}
\tableofcontents{}
}
\pagebreak


\section{Registros de cambios}
\label{sec:registro}


\begin{table}[ht]
\label{tab:registro}
\centering

\begin{tabularx}{\linewidth}{@{}|c|X|c|@{}}
\hline
\rowcolor[HTML]{C0C0C0} 
Revisión & \multicolumn{1}{c|}{\cellcolor[HTML]{C0C0C0}Detalles de los cambios realizados} & Fecha      \\ \hline
1.0      & 	Creación del documento     				& 22/06/2020 \\ \hline
1.1      & 	Se agrego del punto 1 al 6	 			& 05/07/2020 \\ \hline
\end{tabularx}
\end{table}

\pagebreak



\section{Acta de Constitución del Proyecto}
\label{sec:acta}

\begin{flushright}
Buenos Aires, \fechaInicioName
\end{flushright}

\vspace{2 cm}

Por medio de la presente se acuerda con el Ing. \authorname\hspace{1px} que su Trabajo Final de la \degreename\hspace{1px} se titulará ``\ttitle'', consistirá esencialmente en un generador de patrones de descargas parciales, y tendrá un presupuesto preliminar estimado de 600 hs de trabajo, con fecha de inicio \fechaInicioName\hspace{1px} y fecha de presentación pública \fechaFinalName.

Se adjunta a esta acta la planificación inicial.

\vfill

% Esta parte se construye sola con la información que hayan cargado en el preámbulo del documento y no debe modificarla
\begin{table}[ht]
\centering
\begin{tabular}{ccc}
\begin{tabular}[c]{@{}c@{}}Ariel Lutenberg \\ Director posgrado FIUBA\end{tabular} &  & \begin{tabular}[c]{@{}c@{}}\clientename \\ \empclientename \end{tabular} \vspace{2.5cm} \\ 
\begin{tabular}[c]{@{}c@{}}\supname \\ Director del Trabajo Final\end{tabular} &  & 
\begin{tabular}[c]{@{}c@{}}\cosupname \\ Co-Director del Trabajo Final \end{tabular} \vspace{2.5cm} \\ 
\begin{tabular}[c]{@{}c@{}}\jurunoname \\ Jurado del Trabajo Final\end{tabular}     &  & \begin{tabular}[c]{@{}c@{}}\jurdosname\\ Jurado del Trabajo Final\end{tabular}  \vspace{2.5cm}  \\
\multicolumn{3}{c}{\begin{tabular}[c]{@{}c@{}} \jurtresname\\ Jurado del Trabajo Final\end{tabular}} \vspace{.5cm}                                                                     
\end{tabular}
\end{table}




\section{Descripción técnica-conceptual del Proyecto a realizar}
\label{sec:descripcion}

Introducción general al tema

Una descarga parcial es un fenómeno de disrupción eléctrica. Se caracteriza por ser un
pulso de corriente de alta frecuencia el cual se produce en el seno de un sistema aislante de
una máquina o equipo eléctrico de potencia como consecuencia de la presencia de
oclusiones gaseosas, impurezas, aristas aguzadas u otras anomalías que distorsionan la
distribución de las líneas de campo eléctrico.
La ocurrencia de este fenómeno provoca un deterioro local o total del sistema aislante;
dependiendo de la tecnología de aislación utilizada. Indistintamente de cuál sea el medio en
el que este fenómeno se manifiesta y cual sea la causa que lo origina, el deterioro del
sistema es acumulativo.
El detector de descargas parciales es un equipo que permite realizar mediciones de
distintos tipos de DP en máquinas o equipos eléctricos de potencia. Por medio de estas
mediciones es posible monitorear el estado de su aislación.
En la figura 1 se presenta el diagrama en bloques del sistema.

\begin{figure}[htpb]
\centering 
\includegraphics[width=.7\textwidth]{./Figuras/diagBloques1.png}
\caption{Diagrama en bloques del sistema}
\label{fig:diagBloques}
\end{figure}

Descripción detallada

El objetivo del presente trabajo es desarrollar un hardware capaz de detectar los picos
máximos de los pulsos de DP y representarlos sobre una senoide de referencia de
frecuencia industrial - 50 Hertz – en fase con la tensión de ensayo. La superposición de
eventos conformará una nube cuya estructura o morfología dará indicios del tipo de DP
(corona, interna o superficial).
A la representación antes mencionada se la conoce con el nombre de “Patrón de DP”. Una
imagen de este puede apreciarse en la figura número 2.

\begin{figure}[htpb]
\centering 
\includegraphics[width=.3\textwidth]{./Figuras/patronDP.png}
\caption{Diagrama en bloques del sistema}
\label{fig:diagBloques}
\end{figure}

Situación actual

Actualmente existen equipos de medición de descargas parciales fabricados por empresas extranjeras como TechImp. No obstante, debido a su diseño, los equipos existentes en el mercado son equipos de alto costo no aptos para trabajar como equipos estacionarios.

Mediante este proyecto se pretende crear un equipo de bajo costo capaz de ser instalado permanentemente para monitoreo en tiempo real.


\section{Identificación y análisis de los interesados}
\label{sec:interesados}

\begin{table}[ht]
%\caption{Identificación de los interesados}
%\label{tab:interesados}
\begin{tabularx}{\linewidth}{@{}|l|X|X|l|@{}}
\hline
\rowcolor[HTML]{C0C0C0} 
Rol           & Nombre y Apellido & Organización 	& Puesto 	\\ \hline
Cliente       & \clientename      &\empclientename	&        	\\ \hline
Impulsor      & \authorname       & FIUBA         	& Alumno   	\\ \hline
Responsable   & \authorname       & FIUBA        	& Alumno 	\\ \hline
Orientador    & \supname 		  & \pertesupname 	& Director	Trabajo final \\ \hline
Orientador    & \cosupname 		  & \pertesupname 	& Co-Director	Trabajo final \\ \hline
\end{tabularx}
\end{table}


\section{1. Propósito del proyecto}
\label{sec:proposito}

El propósito de este proyecto es crear un equipo de medición de descargas parciales de bajo costo capaz de ser instalado permanentemente, para brindar un monitoreo constante, rápido y eficiente.

\section{2. Alcance del proyecto}
\label{sec:alcance}

El presente proyecto incluye:
\begin{itemize}
\item El desarrollo del prototipo del producto.
\item El desarrollo del firmware.
\item El diseño electrónico del hardware.
\item Confección de un manual de uso.
\item Pruebas de validación y verificación.
\end{itemize}

El presente proyecto no incluye:
\begin{itemize}
\item El desarrollo del equipo final.
\item Diseño de gabinete.
\item Pruebas en campo.
\end{itemize}


\section{3. Supuestos del proyecto}
\label{sec:supuestos}

\begin{itemize}
\item Las características de las señales a utilizar durante la etapa de prueba serán provistas por UTN FRP.
\item En el ámbito del COVID 2019, no se cerraran las importaciones impidiendo el acceso a equipamiento o materiales necesarios para la ejecución del proyecto.
\end{itemize}

\section{4. Requerimientos}
\label{sec:requerimientos}

\begin{enumerate}
\item Requerimientos generales
	\begin{enumerate}
	\item El dispositivo deberá, mediante el procesamiento de las adquisiciones, detectar los picos máximos de los pulsos de DP y representarlos sobre una senoide de referencia de frecuencia industrial - 50 Hertz - en fase con la tensión de ensayo.
	\item El dispositivo deberá funcionar como un sistema "stand-alone".
	\item El dispositivo debe tener capacidad de almacenar al menos 1000 descargas parciales.
	\item El dispositivo debe tener una entrada para la señal analógica y una entrada para la señal de referencia.	
	\end{enumerate}
\item Requerimientos asociados a la adquisición de la señal analógica
	\begin{enumerate}
	\item Se deben poder adquirir señales de al menos 45Mhz con una resolución mínima de 8 bits
	\item El trigger de disparo debe ser configurable.
	\item La duración del pulso medido sera como máximo 1 mili segundo.
	\item La amplitud máxima de la señal de entrada sera de 1Vpp.
	\item La entrada para el sensor analógico deberá ser de 50 ohms diferencial.
	\end{enumerate}
\item Requerimientos asociados a la adquisición de la señal de referencia
	\begin{enumerate}
	\item La entrada de señal de referencia debe poder detectar los cruces por cero de una senoide de 50 Hz.
	\item La entrada de señal de referencia debe ser opto-acoplada.
	\end{enumerate}
\item Requerimientos asociados a la interfaz de usuario
	\begin{enumerate}
	\item Deberá contar con un acceso a un puerto SD o USB para la descarga de datos. 
	\item Deberá contar con un display y teclado para configuración de parámetros básicos.
	\item Deberá poder graficar el patrón de descargas parciales actual en el display.
	\item Deberá tener un puerto de acceso serial (preferentemente diferencial) para configuración y acceso a datos remoto.
	
	\end{enumerate}
\end{enumerate}

\section{5. Entregables principales del proyecto}
\label{sec:entregables}

\begin{itemize}
\item Manual de uso
\item Diagrama esquemático
\item Diagrama de instalación
\item Informe final

\end{itemize}

\section{6. Desglose del trabajo en tareas}
\label{sec:wbs}

\begin{enumerate}
\item Planificación
	\begin{enumerate}
	\item Realizar el plan de proyecto (16hs)
	\end{enumerate}
\item Recopilación de información
	\begin{enumerate}
	\item Búsqueda de información teórica de muestreo a alta velocidad (12hs)
	\item Búsqueda de componentes para la etapa de adquisición analógica (12hs)
	\item Búsqueda de información de puertos de comunicación de alta velocidad (4hs)
	\end{enumerate}
\item Análisis de periféricos de comunicación
	\begin{enumerate}
	\item Codificar funciones para testear limitaciones en la comunicación con el ADC. (20hs)
	\item Codificar funciones para simular la salida digital del ADC a alta velocidad. (20hs)
	\item Codificar funciones para el trigger de disparo. (24hs)
	\end{enumerate}
\item Diseño de hardware
	\begin{enumerate}
	\item Selección y calculo de filtros. (8hs)
	\item Selección de componentes. (20hs)
	\item Diseño del esquemático para la etapa de adquisición analógica. (20hs)
	\item Diseño del PCB para la etapa de adquisición analógica. (12hs)
	\item Montaje de componentes. (8hs)
	\item Verificación y puesta en marcha del hardware. (20hs)
	\item Validación de señales analógicas. (8hs)
	\end{enumerate}
\item Desarrollo modulo de adquisición
	\begin{enumerate}
	\item Codificar funciones de comunicación con ADC. (40hs)
	\item Codificar funciones de detección de cruce por cero. (8hs)
	\item Codificar funciones de integración entre la senoide de referencia y la descarga parcial. (24hs)
	\item Integrar funciones de trigger. (12hs)
	\item Testeo de integridad por medio de inyección de señales simuladas. (24hs)
	\end{enumerate}
\item Desarrollo modulo de comunicación por linea de comandos
	\begin{enumerate}
	\item Codificar funciones de comunicación para la UART. (16hs)
	\end{enumerate}
\item Desarrollo modulo interfaz de usuario
	\begin{enumerate}
	\item Codificar funciones de comunicación con display. (24hs)
	\item Codificar funciones del teclado de entrada. (4hs)
	\item Codificar menú de configuración. (20hs)
	\end{enumerate}
\item Desarrollo modulo de almacenamiento
	\begin{enumerate}
	\item Codificar funciones de almacenamiento USB. (40hs)
	\end{enumerate}
\item Integración
	\begin{enumerate}
	\item Integrar todos los módulos de software. (40hs) 	
	\item Testeo funcional.	(30hs)
	\item Diseño del esquemático final.	(24hs)
	\end{enumerate}
\item Proceso final
	\begin{enumerate}
	\item Confeccionar manual de uso. (10hs) 	
	\item Confeccionar memoria (50hs??)
	\item Revisión final del documento (24hs??)
	\end{enumerate}
\end{enumerate}

Cantidad total de horas: (600 hs)


\section{7. Diagrama de Activity On Node}
\label{sec:AoN}

\begin{consigna}{red}
Armar el AoN a partir del WBS definido en la etapa anterior. 

%La figura \ref{fig:AoN} fue elaborada con el paquete latex tikz y pueden consultar la siguiente referencia \textit{online}:

%\url{https://www.overleaf.com/learn/latex/LaTeX_Graphics_using_TikZ:_A_Tutorial_for_Beginners_(Part_3)\%E2\%80\%94Creating_Flowcharts}

\end{consigna}

\begin{figure}[htpb]
\centering 
\includegraphics[width=.8\textwidth]{./Figuras/AoN.png}
\caption{Diagrama en \textit{Activity on Node}}
\label{fig:AoN}
\end{figure}

Indicar claramente en qué unidades están expresados los tiempos.
De ser necesario indicar los caminos semicríticos y analizar sus tiempos mediante un cuadro.
Es recomendable usar colores y un cuadro indicativo describiendo qué representa cada color, como se muestra en el siguiente ejemplo:



\section{8. Diagrama de Gantt}
\label{sec:gantt}

\begin{consigna}{red}
Utilizar el software Gantter for Google Drive o alguno similar para dibujar el diagrama de Gantt.

Existen muchos programas y recursos \textit{online} para hacer diagramas de gantt, entre las cuales destacamos:

\begin{itemize}
\item Planner
\item GanttProject
\item Trello + \textit{plugins}. En el siguiente link hay un tutorial oficial: \\ \url{https://blog.trello.com/es/diagrama-de-gantt-de-un-proyecto}
\item Creately, herramienta online colaborativa. \\\url{https://creately.com/diagram/example/ieb3p3ml/LaTeX}
\item Se puede hacer en latex con el paquete \textit{pgfgantt}\\ \url{http://ctan.dcc.uchile.cl/graphics/pgf/contrib/pgfgantt/pgfgantt.pdf}
\end{itemize}

Pegar acá una captura de pantalla del diagrama de Gantt, cuidando que la letra sea suficientemente grande como para ser legible. 
Si el diagrama queda demasiado ancho, se puede pegar primero la ``tabla'' del Gantt y luego pegar la parte del diagrama de barras del diagrama de Gantt.

Configurar el software para que en la parte de la tabla muestre los códigos del EDT (WBS).\\
Configurar el software para que al lado de cada barra muestre el nombre de cada tarea.\\
Revisar que la fecha de finalización coincida con lo indicado en el Acta Constitutiva.

En la figura \ref{fig:gantt}, se muestra un ejemplo de diagrama de gantt realizado con el paquete de \textit{pgfgantt}. En la plantilla pueden ver el código que lo genera y usarlo de base para construir el propio.

\begin{figure}[htbp]
\begin{center}
\begin{ganttchart}{1}{12}
  \gantttitle{2020}{12} \\
  \gantttitlelist{1,...,12}{1} \\
  \ganttgroup{Group 1}{1}{7} \\
  \ganttbar{Task 1}{1}{2} \\
  \ganttlinkedbar{Task 2}{3}{7} \ganttnewline
  \ganttmilestone{Milestone o hito}{7} \ganttnewline
  \ganttbar{Final Task}{8}{12}
  \ganttlink{elem2}{elem3}
  \ganttlink{elem3}{elem4}
\end{ganttchart}
\end{center}
\caption{Diagrama de gantt de ejemplo}
\label{fig:gantt}
\end{figure}

\end{consigna}

\section{9. Matriz de uso de recursos de materiales}
\label{sec:recursos}


\begin{table}[htpb]
\label{tab:recursos}
\centering
\begin{tabularx}{\linewidth}{@{}|c|X|X|X|X|X|@{}}
\hline
\cellcolor[HTML]{C0C0C0} & \cellcolor[HTML]{C0C0C0} & \multicolumn{4}{c|}{\cellcolor[HTML]{C0C0C0}Recursos requeridos (horas)} \\ \cline{3-6} 
\multirow{-2}{*}{\cellcolor[HTML]{C0C0C0}\begin{tabular}[c]{@{}c@{}}Código\\ WBS\end{tabular}} & \multirow{-2}{*}{\cellcolor[HTML]{C0C0C0}\begin{tabular}[c]{@{}c@{}}Nombre \\ tarea\end{tabular}} & Material 1 & Material 2 & Material 3 & Material 4 \\ \hline
 &  &  &  &  &  \\ \hline
 &  &  &  &  &  \\ \hline
 &  &  &  &  &  \\ \hline
 &  &  &  &  &  \\ \hline
\end{tabularx}%
\end{table}


\section{10. Presupuesto detallado del proyecto}
\label{sec:presupuesto}

\begin{consigna}{red}
Si el proyecto es complejo entonces separarlo en partes:
\begin{itemize}
\item Un total global, indicando el subtotal acumulado por cada una de las áreas.
\item El desglose detallado del subtotal de cada una de las áreas.
\end{itemize}

IMPORTANTE: No olvidarse de considerar los COSTOS INDIRECTOS.

\end{consigna}

\begin{table}[htpb]
\centering
\begin{tabularx}{\linewidth}{@{}|X|c|r|r|@{}}
\hline
\rowcolor[HTML]{C0C0C0} 
\multicolumn{4}{|c|}{\cellcolor[HTML]{C0C0C0}COSTOS DIRECTOS} \\ \hline
\rowcolor[HTML]{C0C0C0} 
Descripción &
  \multicolumn{1}{c|}{\cellcolor[HTML]{C0C0C0}Cantidad} &
  \multicolumn{1}{c|}{\cellcolor[HTML]{C0C0C0}Valor unitario} &
  \multicolumn{1}{c|}{\cellcolor[HTML]{C0C0C0}Valor total} \\ \hline
 &
  \multicolumn{1}{c|}{} &
  \multicolumn{1}{c|}{} &
  \multicolumn{1}{c|}{} \\ \hline
 &
  \multicolumn{1}{c|}{} &
  \multicolumn{1}{c|}{} &
  \multicolumn{1}{c|}{} \\ \hline
\multicolumn{1}{|l|}{} &
   &
   &
   \\ \hline
\multicolumn{1}{|l|}{} &
   &
   &
   \\ \hline
\multicolumn{3}{|c|}{SUBTOTAL} &
  \multicolumn{1}{c|}{} \\ \hline
\rowcolor[HTML]{C0C0C0} 
\multicolumn{4}{|c|}{\cellcolor[HTML]{C0C0C0}COSTOS INDIRECTOS} \\ \hline
\rowcolor[HTML]{C0C0C0} 
Descripción &
  \multicolumn{1}{c|}{\cellcolor[HTML]{C0C0C0}Cantidad} &
  \multicolumn{1}{c|}{\cellcolor[HTML]{C0C0C0}Valor unitario} &
  \multicolumn{1}{c|}{\cellcolor[HTML]{C0C0C0}Valor total} \\ \hline
\multicolumn{1}{|l|}{} &
   &
   &
   \\ \hline
\multicolumn{1}{|l|}{} &
   &
   &
   \\ \hline
\multicolumn{1}{|l|}{} &
   &
   &
   \\ \hline
\multicolumn{3}{|c|}{SUBTOTAL} &
  \multicolumn{1}{c|}{} \\ \hline
\rowcolor[HTML]{C0C0C0}
\multicolumn{3}{|c|}{TOTAL} &
   \\ \hline
\end{tabularx}%
\end{table}


\section{11. Matriz de asignación de responsabilidades}
\label{sec:responsabilidades}
\begin{consigna}{red}
Establecer la matriz de asignación de responsabilidades y el manejo de la autoridad completando la siguiente tabla:

\begin{table}[htpb]
\centering
\resizebox{\textwidth}{!}{%
\begin{tabular}{|c|c|c|c|c|c|}
\hline
\rowcolor[HTML]{C0C0C0} 
\cellcolor[HTML]{C0C0C0} &
  \cellcolor[HTML]{C0C0C0} &
  \multicolumn{4}{c|}{\cellcolor[HTML]{C0C0C0}Listar todos los nombres y roles del proyecto} \\ \cline{3-6} 
\rowcolor[HTML]{C0C0C0} 
\cellcolor[HTML]{C0C0C0} &
  \cellcolor[HTML]{C0C0C0} &
  Responsable &
  Orientador &
  Equipo &
  Cliente \\ \cline{3-6} 
\rowcolor[HTML]{C0C0C0} 
\multirow{-3}{*}{\cellcolor[HTML]{C0C0C0}\begin{tabular}[c]{@{}c@{}}Código\\ WBS\end{tabular}} &
  \multirow{-3}{*}{\cellcolor[HTML]{C0C0C0}Nombre de la tarea} &
  \authorname &
  \supname &
  Nombre de alguien &
  \clientename \\ \hline
 &  &  &  &  &  \\ \hline
 &  &  &  &  &  \\ \hline
 &  &  &  &  &  \\ \hline
\end{tabular}%
}
\end{table}

{\footnotesize
Referencias:
\begin{itemize}
	\item P = Responsabilidad Primaria
	\item S = Responsabilidad Secundaria
	\item A = Aprobación
	\item I = Informado
	\item C = Consultado
\end{itemize}
} %footnotesize

Una de las columnas debe ser para el Director, ya que se supone que participará en el proyecto.
A su vez se debe cuidar que no queden muchas tareas seguidas sin ``A'' o ``I''.

Importante: es redundante poner ``I/A'' o ``I/C'', porque para aprobarlo o responder consultas primero la persona debe ser informada.

\end{consigna}

\section{12. Gestión de riesgos}
\label{sec:riesgos}

\begin{consigna}{red}
a) Identificación de los riesgos (al menos cinco) y estimación de sus consecuencias:
 
Riesgo 1: detallar el riesgo (riesgo es algo que si ocurre altera los planes previstos)
\begin{itemize}
\item Severidad (S): mientras más severo, más alto es el número (usar números del 1 al 10).\\
Justificar el motivo por el cual se asigna determinado número de severidad (S).
\item Probabilidad de ocurrencia (O): mientras más probable, más alto es el número (usar del 1 al 10).\\
Justificar el motivo por el cual se asigna determinado número de (O). 
\end{itemize}   

Riesgo 2:
\begin{itemize}
\item Severidad (S): 
\item Ocurrencia (O):
\end{itemize}

Riesgo 3:
\begin{itemize}
\item Severidad (S): 
\item Ocurrencia (O):
\end{itemize}


b) Tabla de gestión de riesgos:      (El RPN se calcula como RPN=SxO)

\begin{table}[htpb]
\centering
\begin{tabularx}{\linewidth}{@{}|X|c|c|c|c|c|c|@{}}
\hline
\rowcolor[HTML]{C0C0C0} 
Riesgo & S & O & RPN & S* & O* & RPN* \\ \hline
       &   &   &     &    &    &      \\ \hline
       &   &   &     &    &    &      \\ \hline
       &   &   &     &    &    &      \\ \hline
       &   &   &     &    &    &      \\ \hline
       &   &   &     &    &    &      \\ \hline
\end{tabularx}%
\end{table}

Criterio adoptado: 
Se tomarán medidas de mitigación en los riesgos cuyos números de RPN sean mayores a ....

Nota: los valores marcados con (*) en la tabla corresponden luego de haber aplicado la mitigación.

c) Plan de mitigación de los riesgos que originalmente excedían el RPN máximo establecido:
 
Riesgo 1: Plan de mitigación (si por el RPN fuera necesario elaborar un plan de mitigación).
  Nueva asignación de S y O, con su respectiva justificación:
  - Severidad (S): mientras más severo, más alto es el número (usar números del 1 al 10).
          Justificar el motivo por el cual se asigna determinado número de severidad (S).
  - Probabilidad de ocurrencia (O): mientras más probable, más alto es el número (usar del 1 al 10).
          Justificar el motivo por el cual se asigna determinado número de (O).

Riesgo 2: Plan de mitigación (si por el RPN fuera necesario elaborar un plan de mitigación).
 
Riesgo 3: Plan de mitigación (si por el RPN fuera necesario elaborar un plan de mitigación)

\end{consigna}


\section{13. Gestión de la calidad}
\label{sec:calidad}

\begin{consigna}{red}
Para cada uno de los requerimientos del proyecto indique:
\begin{itemize} 
\item Req \#1: Copiar acá el requerimiento.

Verificación y validación:

\begin{itemize}
\item Verificación para confirmar si se cumplió con lo requerido antes de mostrar el sistema al cliente:\\
Detallar 
\item Validación con el cliente para confirmar que está de acuerdo en que se cumplió con lo requerido:\\
Detallar  
\end{itemize}

\end{itemize}

Tener en cuenta que en este contexto se pueden mencionar simulaciones, cálculos, revisión de hojas de datos, consulta con expertos, etc.

\end{consigna}

\section{14. Comunicación del proyecto}
\label{sec:comunicaciones}

\begin{consigna}{red}
El plan de comunicación del proyecto es el siguiente:
\end{consigna}

% Please add the following required packages to your document preamble:
% \usepackage{graphicx}
% \usepackage[table,xcdraw]{xcolor}
% If you use beamer only pass "xcolor=table" option, i.e. \documentclass[xcolor=table]{beamer}
\begin{table}[htpb]
\centering
\resizebox{\textwidth}{!}{%
\begin{tabular}{|c|c|c|c|c|c|}
\hline
\rowcolor[HTML]{C0C0C0} 
\multicolumn{6}{|c|}{\cellcolor[HTML]{C0C0C0}PLAN DE COMUNICACIÓN DEL PROYECTO}           \\ \hline
\rowcolor[HTML]{C0C0C0} 
¿Qué comunicar? & Audiencia & Propósito & Frecuencia & Método de comunicac. & Responsable \\ \hline
                &           &           &            &                      &             \\ \hline
                &           &           &            &                      &             \\ \hline
                &           &           &            &                      &             \\ \hline
                &           &           &            &                      &             \\ \hline
                &           &           &            &                      &             \\ \hline
\end{tabular}%
}
\end{table}

\section{15. Gestión de Compras}
\label{sec:compras}

\begin{consigna}{red}
En caso de tener que comprar elementos o contratar servicios:
a) Explique con qué criterios elegiría a un proveedor.
b) Redacte el Statement of Work correspondiente.
\end{consigna}

\section{16. Seguimiento y control}
\label{sec:seguimiento}

\begin{consigna}{red}
Para cada tarea del proyecto establecer la frecuencia y los indicadores con los se seguirá su avance y quién será el responsable de hacer dicho seguimiento y a quién debe comunicarse la situación (en concordancia con el Plan de Comunicación del proyecto).

El indicador de avance tiene que ser algo medible, mejor incluso si se puede medir en \% de avance. Por ejemplo,se pueden indicar en esta columna cosas como ``cantidad de conexiones ruteadeas'' o ``cantidad de funciones implementadas'', pero no algo genérico y ambiguo como ``\%'', porque el lector no sabe porcentaje de qué cosa.

\end{consigna}

\begin{table}[!htpb]
\centering
\begin{tabularx}{\linewidth}{@{}|X|X|X|X|X|X|@{}}
\hline
\rowcolor[HTML]{C0C0C0} 
\multicolumn{6}{|c|}{\cellcolor[HTML]{C0C0C0}SEGUIMIENTO DE AVANCE}                                                                       \\ \hline
\rowcolor[HTML]{C0C0C0} 
Tarea del WBS & Indicador de avance & Frecuencia de reporte & Resp. de seguimiento & Persona a ser informada & Método de comunic. \\ \hline
 &  &  &  &  &  \\ \hline
 &  &  &  &  &  \\ \hline
 &  &  &  &  &  \\ \hline
 &  &  &  &  &  \\ \hline
 &  &  &  &  &  \\ \hline
\end{tabularx}%
%}
\end{table}

\section{17. Procesos de cierre}    
\label{sec:cierre}

\begin{consigna}{red}
Establecer las pautas de trabajo para realizar una reunión final de evaluación del proyecto, tal que contemple las siguientes actividades:

\begin{itemize}
\item Pautas de trabajo que se seguirán para analizar si se respetó el Plan de Proyecto original:
 - Indicar quién se ocupará de hacer esto y cuál será el procedimiento a aplicar. 
\item Identificación de las técnicas y procedimientos útiles e inútiles que se utilizaron, y los problemas que surgieron y cómo se solucionaron:
 - Indicar quién se ocupará de hacer esto y cuál será el procedimiento para dejar registro.
\item Indicar quién organizará el acto de agradecimiento a todos los interesados, y en especial al equipo de trabajo y colaboradores:
  - Indicar esto y quién financiará los gastos correspondientes.
\end{itemize}

\end{consigna}


\end{document}
